\documentclass[conference]{IEEEtran}
\IEEEoverridecommandlockouts
% The preceding line is only needed to identify funding in the first footnote. If that is unneeded, please comment it out.
\usepackage{cite}
\usepackage{amsmath,amssymb,amsfonts}
\usepackage{algorithmic}
\usepackage{graphicx}
\usepackage{textcomp}
\usepackage{xcolor}
\usepackage[lined,algonl,boxed]{algorithm2e}


\def\BibTeX{{\rm B\kern-.05em{\sc i\kern-.025em b}\kern-.08em
    T\kern-.1667em\lower.7ex\hbox{E}\kern-.125emX}}
\begin{document}




\title{ELEN4020: Data Intensive Computing\\ Lab 3\\
{\footnotesize School of Electrical \& Information Engineering, University of the
Witwatersrand, Private Bag 3, 2050, Johannesburg, South Africa}
%\thanks{Identify applicable funding agency here. If none, delete this.}
}


\author{

\IEEEauthorblockN{Elias Sepuru 1427726}
\and
\IEEEauthorblockN{Boikanyo Radiokana 1386807}
\and
\IEEEauthorblockN{Lloyd Patsika 1041888}

}

\maketitle

\section{INTRODUCTION}
In this report the MapReduce framework is used to count the occurrence of each word in a text file. On top of finding the occurrence of each word, the top K, where K=10,20, occurring words are distinguished. The framework is also used to find the indices of words in a text file. The time taken to count the occurrences of each word in a text file and the indices the words appear on are recorded. Mrs-MapReduce, a lightweight implementation of MapReduce is used. 



%%%%%%%%%%%%%%%%%%%%%%%%%%%%%%%%%%%%%%%%%%%%%%%%%%%%%%%%%%%%%%%%%%%%%%%%%%%%%%%
%
\section{DESIGN \& IMPLEMENTATION}

\linesnumbered
\begin{algorithm}

\centring

\SetKwData{tempSum}{tempSum}
\SetKwData{C}{C}
\SetKwData{tempVector}{tempVector}
\SetKwInOut{Input}{input}
\SetKwInOut{Output}{output}


\caption{\texttt{rank2DTenosrMult(A,B)}: 2D Matrix Multplication}\Input{Two 2D matrices of sizes $n\times n$}
\Output{A resultant C 2D matrix}

\For{$i\gets1$ \KwTo $n$}{

     \For{$j\gets1$ \KwTo $n$}{

	\For{$k\gets1$ \KWTo $n$}{

	     \tempSum$\leftarrow$ \tempSum + {$A[i,k]\times B[j,k]$}\;
	}
	\tempVector$\leftarrow$ \tempSum\;
}

\C$\leftarrow$ \tempVector\;

}
\Return \C
\label{mult2d}
\end{algorithm}



%\begin{thebibliography}{00}
%\bibitem{b1} G. Eason, B. Noble, and I. N. Sneddon, ``On certain integrals of Lipschitz-Hankel type involving products of Bessel functions,'' Phil. Trans. Roy. Soc. London, vol. A247, pp. 529--551, April 1955.

%\end{thebibliography}



\end{document}
